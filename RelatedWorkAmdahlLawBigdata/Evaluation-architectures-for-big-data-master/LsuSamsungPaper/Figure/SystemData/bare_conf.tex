%% bare_conf.tex
%% V1.3
%% 2007/01/11
%% by Michael Shell
%% See:
%% http://www.michaelshell.org/
%% for current contact information.
%%
%% This is a skeleton file demonstrating the use of IEEEtran.cls
%% (requires IEEEtran.cls version 1.7 or later) with an IEEE conference paper.
%%
%% Support sites:
%% http://www.michaelshell.org/tex/ieeetran/
%% http://www.ctan.org/tex-archive/macros/latex/contrib/IEEEtran/
%% and
%% http://www.ieee.org/

%%*************************************************************************
%% Legal Notice:
%% This code is offered as-is without any warranty either expressed or
%% implied; without even the implied warranty of MERCHANTABILITY or
%% FITNESS FOR A PARTICULAR PURPOSE! 
%% User assumes all risk.
%% In no event shall IEEE or any contributor to this code be liable for
%% any damages or losses, including, but not limited to, incidental,
%% consequential, or any other damages, resulting from the use or misuse
%% of any information contained here.
%%
%% All comments are the opinions of their respective authors and are not
%% necessarily endorsed by the IEEE.
%%
%% This work is distributed under the LaTeX Project Public License (LPPL)
%% ( http://www.latex-project.org/ ) version 1.3, and may be freely used,
%% distributed and modified. A copy of the LPPL, version 1.3, is included
%% in the base LaTeX documentation of all distributions of LaTeX released
%% 2003/12/01 or later.
%% Retain all contribution notices and credits.
%% ** Modified files should be clearly indicated as such, including  **
%% ** renaming them and changing author support contact information. **
%%
%% File list of work: IEEEtran.cls, IEEEtran_HOWTO.pdf, bare_adv.tex,
%%                    bare_conf.tex, bare_jrnl.tex, bare_jrnl_compsoc.tex
%%*************************************************************************

% *** Authors should verify (and, if needed, correct) their LaTeX system  ***
% *** with the testflow diagnostic prior to trusting their LaTeX platform ***
% *** with production work. IEEE's font choices can trigger bugs that do  ***
% *** not appear when using other class files.                            ***
% The testflow support page is at:
% http://www.michaelshell.org/tex/testflow/



% Note that the a4paper option is mainly intended so that authors in
% countries using A4 can easily print to A4 and see how their papers will
% look in print - the typesetting of the document will not typically be
% affected with changes in paper size (but the bottom and side margins will).
% Use the testflow package mentioned above to verify correct handling of
% both paper sizes by the user's LaTeX system.
%
% Also note that the "draftcls" or "draftclsnofoot", not "draft", option
% should be used if it is desired that the figures are to be displayed in
% draft mode.
%
\documentclass[conference]{IEEEtran}
% Add the compsoc option for Computer Society conferences.
%
% If IEEEtran.cls has not been installed into the LaTeX system files,
% manually specify the path to it like:
% \documentclass[conference]{../sty/IEEEtran}





% Some very useful LaTeX packages include:
% (uncomment the ones you want to load)


% *** MISC UTILITY PACKAGES ***
%
%\usepackage{ifpdf}
% Heiko Oberdiek's ifpdf.sty is very useful if you need conditional
% compilation based on whether the output is pdf or dvi.
% usage:
% \ifpdf
%   % pdf code
% \else
%   % dvi code
% \fi
% The latest version of ifpdf.sty can be obtained from:
% http://www.ctan.org/tex-archive/macros/latex/contrib/oberdiek/
% Also, note that IEEEtran.cls V1.7 and later provides a builtin
% \ifCLASSINFOpdf conditional that works the same way.
% When switching from latex to pdflatex and vice-versa, the compiler may
% have to be run twice to clear warning/error messages.






% *** CITATION PACKAGES ***
%
%\usepackage{cite}
% cite.sty was written by Donald Arseneau
% V1.6 and later of IEEEtran pre-defines the format of the cite.sty package
% \cite{} output to follow that of IEEE. Loading the cite package will
% result in citation numbers being automatically sorted and properly
% "compressed/ranged". e.g., [1], [9], [2], [7], [5], [6] without using
% cite.sty will become [1], [2], [5]--[7], [9] using cite.sty. cite.sty's
% \cite will automatically add leading space, if needed. Use cite.sty's
% noadjust option (cite.sty V3.8 and later) if you want to turn this off.
% cite.sty is already installed on most LaTeX systems. Be sure and use
% version 4.0 (2003-05-27) and later if using hyperref.sty. cite.sty does
% not currently provide for hyperlinked citations.
% The latest version can be obtained at:
% http://www.ctan.org/tex-archive/macros/latex/contrib/cite/
% The documentation is contained in the cite.sty file itself.






% *** GRAPHICS RELATED PACKAGES ***
%
\ifCLASSINFOpdf
  % \usepackage[pdftex]{graphicx}
  % declare the path(s) where your graphic files are
  % \graphicspath{{../pdf/}{../jpeg/}}
  % and their extensions so you won't have to specify these with
  % every instance of \includegraphics
  % \DeclareGraphicsExtensions{.pdf,.jpeg,.png}
\else
  % or other class option (dvipsone, dvipdf, if not using dvips). graphicx
  % will default to the driver specified in the system graphics.cfg if no
  % driver is specified.
  % \usepackage[dvips]{graphicx}
  % declare the path(s) where your graphic files are
  % \graphicspath{{../eps/}}
  % and their extensions so you won't have to specify these with
  % every instance of \includegraphics
  % \DeclareGraphicsExtensions{.eps}
\fi
% graphicx was written by David Carlisle and Sebastian Rahtz. It is
% required if you want graphics, photos, etc. graphicx.sty is already
% installed on most LaTeX systems. The latest version and documentation can
% be obtained at: 
% http://www.ctan.org/tex-archive/macros/latex/required/graphics/
% Another good source of documentation is "Using Imported Graphics in
% LaTeX2e" by Keith Reckdahl which can be found as epslatex.ps or
% epslatex.pdf at: http://www.ctan.org/tex-archive/info/
%
% latex, and pdflatex in dvi mode, support graphics in encapsulated
% postscript (.eps) format. pdflatex in pdf mode supports graphics
% in .pdf, .jpeg, .png and .mps (metapost) formats. Users should ensure
% that all non-photo figures use a vector format (.eps, .pdf, .mps) and
% not a bitmapped formats (.jpeg, .png). IEEE frowns on bitmapped formats
% which can result in "jaggedy"/blurry rendering of lines and letters as
% well as large increases in file sizes.
%
% You can find documentation about the pdfTeX application at:
% http://www.tug.org/applications/pdftex
\usepackage{graphicx}
\usepackage{placeins}
\usepackage{float}
\usepackage{subfigure}




% *** MATH PACKAGES ***
%
%\usepackage[cmex10]{amsmath}
% A popular package from the American Mathematical Society that provides
% many useful and powerful commands for dealing with mathematics. If using
% it, be sure to load this package with the cmex10 option to ensure that
% only type 1 fonts will utilized at all point sizes. Without this option,
% it is possible that some math symbols, particularly those within
% footnotes, will be rendered in bitmap form which will result in a
% document that can not be IEEE Xplore compliant!
%
% Also, note that the amsmath package sets \interdisplaylinepenalty to 10000
% thus preventing page breaks from occurring within multiline equations. Use:
%\interdisplaylinepenalty=2500
% after loading amsmath to restore such page breaks as IEEEtran.cls normally
% does. amsmath.sty is already installed on most LaTeX systems. The latest
% version and documentation can be obtained at:
% http://www.ctan.org/tex-archive/macros/latex/required/amslatex/math/





% *** SPECIALIZED LIST PACKAGES ***
%
%\usepackage{algorithmic}
% algorithmic.sty was written by Peter Williams and Rogerio Brito.
% This package provides an algorithmic environment fo describing algorithms.
% You can use the algorithmic environment in-text or within a figure
% environment to provide for a floating algorithm. Do NOT use the algorithm
% floating environment provided by algorithm.sty (by the same authors) or
% algorithm2e.sty (by Christophe Fiorio) as IEEE does not use dedicated
% algorithm float types and packages that provide these will not provide
% correct IEEE style captions. The latest version and documentation of
% algorithmic.sty can be obtained at:
% http://www.ctan.org/tex-archive/macros/latex/contrib/algorithms/
% There is also a support site at:
% http://algorithms.berlios.de/index.html
% Also of interest may be the (relatively newer and more customizable)
% algorithmicx.sty package by Szasz Janos:
% http://www.ctan.org/tex-archive/macros/latex/contrib/algorithmicx/




% *** ALIGNMENT PACKAGES ***
%
%\usepackage{array}
% Frank Mittelbach's and David Carlisle's array.sty patches and improves
% the standard LaTeX2e array and tabular environments to provide better
% appearance and additional user controls. As the default LaTeX2e table
% generation code is lacking to the point of almost being broken with
% respect to the quality of the end results, all users are strongly
% advised to use an enhanced (at the very least that provided by array.sty)
% set of table tools. array.sty is already installed on most systems. The
% latest version and documentation can be obtained at:
% http://www.ctan.org/tex-archive/macros/latex/required/tools/


%\usepackage{mdwmath}
%\usepackage{mdwtab}
% Also highly recommended is Mark Wooding's extremely powerful MDW tools,
% especially mdwmath.sty and mdwtab.sty which are used to format equations
% and tables, respectively. The MDWtools set is already installed on most
% LaTeX systems. The lastest version and documentation is available at:
% http://www.ctan.org/tex-archive/macros/latex/contrib/mdwtools/


% IEEEtran contains the IEEEeqnarray family of commands that can be used to
% generate multiline equations as well as matrices, tables, etc., of high
% quality.


%\usepackage{eqparbox}
% Also of notable interest is Scott Pakin's eqparbox package for creating
% (automatically sized) equal width boxes - aka "natural width parboxes".
% Available at:
% http://www.ctan.org/tex-archive/macros/latex/contrib/eqparbox/





% *** SUBFIGURE PACKAGES ***
%\usepackage[tight,footnotesize]{subfigure}
% subfigure.sty was written by Steven Douglas Cochran. This package makes it
% easy to put subfigures in your figures. e.g., "Figure 1a and 1b". For IEEE
% work, it is a good idea to load it with the tight package option to reduce
% the amount of white space around the subfigures. subfigure.sty is already
% installed on most LaTeX systems. The latest version and documentation can
% be obtained at:
% http://www.ctan.org/tex-archive/obsolete/macros/latex/contrib/subfigure/
% subfigure.sty has been superceeded by subfig.sty.



%\usepackage[caption=false]{caption}
%\usepackage[font=footnotesize]{subfig}
% subfig.sty, also written by Steven Douglas Cochran, is the modern
% replacement for subfigure.sty. However, subfig.sty requires and
% automatically loads Axel Sommerfeldt's caption.sty which will override
% IEEEtran.cls handling of captions and this will result in nonIEEE style
% figure/table captions. To prevent this problem, be sure and preload
% caption.sty with its "caption=false" package option. This is will preserve
% IEEEtran.cls handing of captions. Version 1.3 (2005/06/28) and later 
% (recommended due to many improvements over 1.2) of subfig.sty supports
% the caption=false option directly:
%\usepackage[caption=false,font=footnotesize]{subfig}
%
% The latest version and documentation can be obtained at:
% http://www.ctan.org/tex-archive/macros/latex/contrib/subfig/
% The latest version and documentation of caption.sty can be obtained at:
% http://www.ctan.org/tex-archive/macros/latex/contrib/caption/




% *** FLOAT PACKAGES ***
%
%\usepackage{fixltx2e}
% fixltx2e, the successor to the earlier fix2col.sty, was written by
% Frank Mittelbach and David Carlisle. This package corrects a few problems
% in the LaTeX2e kernel, the most notable of which is that in current
% LaTeX2e releases, the ordering of single and double column floats is not
% guaranteed to be preserved. Thus, an unpatched LaTeX2e can allow a
% single column figure to be placed prior to an earlier double column
% figure. The latest version and documentation can be found at:
% http://www.ctan.org/tex-archive/macros/latex/base/



%\usepackage{stfloats}
% stfloats.sty was written by Sigitas Tolusis. This package gives LaTeX2e
% the ability to do double column floats at the bottom of the page as well
% as the top. (e.g., "\begin{figure*}[!b]" is not normally possible in
% LaTeX2e). It also provides a command:
%\fnbelowfloat
% to enable the placement of footnotes below bottom floats (the standard
% LaTeX2e kernel puts them above bottom floats). This is an invasive package
% which rewrites many portions of the LaTeX2e float routines. It may not work
% with other packages that modify the LaTeX2e float routines. The latest
% version and documentation can be obtained at:
% http://www.ctan.org/tex-archive/macros/latex/contrib/sttools/
% Documentation is contained in the stfloats.sty comments as well as in the
% presfull.pdf file. Do not use the stfloats baselinefloat ability as IEEE
% does not allow \baselineskip to stretch. Authors submitting work to the
% IEEE should note that IEEE rarely uses double column equations and
% that authors should try to avoid such use. Do not be tempted to use the
% cuted.sty or midfloat.sty packages (also by Sigitas Tolusis) as IEEE does
% not format its papers in such ways.





% *** PDF, URL AND HYPERLINK PACKAGES ***
%
%\usepackage{url}
% url.sty was written by Donald Arseneau. It provides better support for
% handling and breaking URLs. url.sty is already installed on most LaTeX
% systems. The latest version can be obtained at:
% http://www.ctan.org/tex-archive/macros/latex/contrib/misc/
% Read the url.sty source comments for usage information. Basically,
% \url{my_url_here}.

\usepackage{paralist}



% *** Do not adjust lengths that control margins, column widths, etc. ***
% *** Do not use packages that alter fonts (such as pslatex).         ***
% There should be no need to do such things with IEEEtran.cls V1.6 and later.
% (Unless specifically asked to do so by the journal or conference you plan
% to submit to, of course. )


% correct bad hyphenation here
\hyphenation{op-tical net-works semi-conduc-tor}


\begin{document}
%
% paper title
% can use linebreaks \\ within to get better formatting as desired
\title{Scaleout to Scaleup: From Traditional Supercomputer to NextGen Bigcomputer for Data-intensive Scientific Applications}


% author names and affiliations
% use a multiple column layout for up to three different
% affiliations
\author{\IEEEauthorblockN{Arghya Kusum Das, Seung-Jong Park}
\IEEEauthorblockA{School of Electrical Engineering and Computer Science\\
Center for Computation and Technology\\
Louisisna State University\\
Baton Rouge, LA, 70801 \\
Email: \{adas7, sjpark\} @lsu.edu}
%\and
%\IEEEauthorblockN{Homer Simpson}
%\IEEEauthorblockA{Twentieth Century Fox\\
%Springfield, USA\\
%Email: homer@thesimpsons.com}
%\and
%\IEEEauthorblockN{James Kirk\\ and Montgomery Scott}
%\IEEEauthorblockA{Starfleet Academy\\
%San Francisco, California 96678-2391\\
%Telephone: (800) 555--1212\\
%Fax: (888) 555--1212}
}

% conference papers do not typically use \thanks and this command
% is locked out in conference mode. If really needed, such as for
% the acknowledgment of grants, issue a \IEEEoverridecommandlockouts
% after \documentclass

% for over three affiliations, or if they all won't fit within the width
% of the page, use this alternative format:
% 
%\author{\IEEEauthorblockN{Michael Shell\IEEEauthorrefmark{1},
%Homer Simpson\IEEEauthorrefmark{2},
%James Kirk\IEEEauthorrefmark{3}, 
%Montgomery Scott\IEEEauthorrefmark{3} and
%Eldon Tyrell\IEEEauthorrefmark{4}}
%\IEEEauthorblockA{\IEEEauthorrefmark{1}School of Electrical and Computer Engineering\\
%Georgia Institute of Technology,
%Atlanta, Georgia 30332--0250\\ Email: see http://www.michaelshell.org/contact.html}
%\IEEEauthorblockA{\IEEEauthorrefmark{2}Twentieth Century Fox, Springfield, USA\\
%Email: homer@thesimpsons.com}
%\IEEEauthorblockA{\IEEEauthorrefmark{3}Starfleet Academy, San Francisco, California 96678-2391\\
%Telephone: (800) 555--1212, Fax: (888) 555--1212}
%\IEEEauthorblockA{\IEEEauthorrefmark{4}Tyrell Inc., 123 Replicant Street, Los Angeles, California 90210--4321}}




% use for special paper notices
%\IEEEspecialpapernotice{(Invited Paper)}




% make the title area
\maketitle


\begin{abstract}
%Efficient processing of bigdata produced by different scientific experimental facilities poses several challenges in terms of efficient storage, transfer, in-memory computation and exploiting locality of these data. 
%The existing supercomputers being optimally tuned for high performance compute intensive applications like MPI, Grid etc cannot address most of these challenges especially due to their storage and memory constraint.
%On the other hand, in the last decade, Hadoop and other state of the art bigdata analytics softwares shifted the existing model of computation severely towards data driven analysis creating new oportunity to address many HPC challenges involving bigdata.
%Although, these massively parallel bigdata analytics softwares made several HPC computation easy, efficient provisioning in terms hardwares still remain a major challenge especially in the context of data intensive high performance computation.

%In this paper we juxtapose different distributed cyber infrastructure including traditional supercomputers and available cloud resources with Intel hiBench and our bigdata Parallel Genome Assembler (PGA) based upon Hadoop and Giraph, that serves as a very good example of data as well as compute intensive application.
%We compare the perfromance of LSU supercomputer QueenBeeII and two different types of clusters located in Samsung Korea with our assembler and observed Samsung Cluster with [SSD and 10GigE....] outperformed QueenBeeII in terms of [price to performance....] because of [high processor density, higher IO throughput....].
%Unlike other studies, our analysis is not restricted in comparing any single system component (eg. storage or network interconnection) rather motivated to answer a higher level question that is becoming increasingly important 'how does the next generation bigdata (tera/petabyte scale) high performance computation cluster should look like'
The enormous growth of bigdata produced by different experimental facilities is rapidly changing the model of computation in the domain of high performance computing (HPC).
Many HPC aficionados, in order to efficiently manage their data intensive workload started using the current state of the bigdata analytics softwares like Hdoop, giraph etc. devieting from traditional parallel programming models like MPI etc.
%Unprecedented amount of bigdata produced by different advanced scientific experimental facilities make the high performance computation a severe data intensive endeavor.
%In order to address the challnges involved in efficient storage, management and analysis of these bigdata, recently, many HPC aficionados started using the current state of the art bigdata analytics softwares like Hadoop, Giraph etc for their data intensive scientific workloads diviating from traditional compute-intensive programming paradigm like MPI, Grid etc.
Consequently, traditional supercomputers with lots of processing power are found to provide suboptimal performance due to several limitations in the underlying hardware infrastructure creating new opportunities for hardware manufacturers as well as the cloud service providers.

In this paper, we compare the performance of a traditional suprcomputer named SuperMikeII located in LSU and our state of the art bigdata analytics cluster called SwatIII located in Samsung, Korea) while handling a data-intensive workload. and show how we achieve x-speedup using only 1/x nodes and subsequently x\% performance to price benefit in SwatIII.
%There is limited understanding on the performance characteristics of the state of the art bigdata analytics softwares on different type of distributed cyberinfrastructure when applied for data intensive scientific workload.
Our analysis is based upon our benchmark large-scale Parallel Genome Assembler (PGA) based upon Hadoop and Giraph. 
Our assembly pipeline consists of a huge amount of short read analysis using Hadoop, followed by a large de Bruijn graph analysis using Giraph, thus serving as a very good real world example of data as well as compute intensive workload.
%We evaluate the performance of our assembler atop LSU supercomputer, SuperMikeII and two different types of state of the art bigdata analytics clusters, SwatIII and CeresII located in Samsung Korea. 
%Our analysis provides insights on the actual limitations of tradional supercomputers as well as pinpoint various design considerations and performance-tradeoffs from the perspective of efficient hardware provisioning.
%We show how we achieve almost x-speedup and y\% performance to price benefit in our state of the art cluster than a traditional supercomputer.
\end{abstract}
% IEEEtran.cls defaults to using nonbold math in the Abstract.
% This preserves the distinction between vectors and scalars. However,
% if the conference you are submitting to favors bold math in the abstract,
% then you can use LaTeX's standard command \boldmath at the very start
% of the abstract to achieve this. Many IEEE journals/conferences frown on
% math in the abstract anyway.

% no keywords

% For peer review papers, you can put extra information on the cover
% page as needed:
% \ifCLASSOPTIONpeerreview
% \begin{center} \bfseries EDICS Category: 3-BBND \end{center}
% \fi
%
% For peerreview papers, this IEEEtran command inserts a page break and
% creates the second title. It will be ignored for other modes.
\IEEEpeerreviewmaketitle



\section {Introduction}
Scientists in different fields are increasingly handling huge amount of bigdata produced by different experimental facilites which make the so called compute intensive scientific applications a severe data intensive endeavor. 
Starting from astronomical data analysis to coastal simulation, from social data analysis to genome assembly, this huge volume of data poses several challneges from effectively storing and managing to optimally processing it.
The fundamental model of computation involved in the scientific applications is rapidly changing in order to address these challenges.
Deviating from the decade old compute intensive programming paradigm like MPI, Grid etc. many HPC aficionados has started using the current state of the art big data analytics software like Hadoop, Giraph etc. for their data intensive scientific workloads.

Consequently, the traditional supercomputers even with tera to peta FLOP scale processing power are found to yield suboptimal performance concluding the processing power is not the only factor of actual performance for these data intensive workloads.
The cumulative effect of CPU, memory, disk and network architecture on the over all performance make the job of providing efficient yet cost-effective hardwares more challenging, however, opening new opportunities for the hardware manufacturers.
As a result, there is a growing interest in both HPC community as well as the hardware companies to address the challenges involved in providing cost-efective high performance testbeds that will drive the next generation data intensive scientific research.
Furthermore, in the last few years, the scientific community has also experienced several benefits of pay-as-you-go cloud services (eg. Amazon-Cloud, Penguin, R-HPC etc) including the  elasticity of resources with reduced setup cost and time which also has a catalytic effect on this interest.
As a consequence, commercial cloud service providers started investing a lot to update and upgrade their infrastructures.
Also, millions of dollars are being spent in programs like NSFCloud where several academic organizations and manufacturing companies collaborated to enable the academic research community to develop and experiment with novel cloud architectures.

Despite of this growing interest in both scientific as well as industrial community, there is very limited understanding of the performance characteristics of the current state of the art bigdata analytics softwares when applied for high performance data intensive scientific workloads on different hardware infrastructure. 
Thus, we found it extremely important to evaluate different types of distributed cyber infrastructure to understand the limitation in traditional supercomputers as well as the impact of different types of storage media, networking technologies and the overall cluster architecture and organization on the state of the art bigdata analytics softwrares in the context of a real world data intensive high performance complex scientific workload.

In this work, we use large scale de novo genome assembly as one of the most challenging and complex real world example of high performance computing that recently made its way to the forefront of bigdata challenges.
De novo genome assembly reconstructs the entire genome from fragmented parts called short reads when no reference genome is available.
The assembly pipeline consists of huge amount of short read analysis followed by a  complex analysis on a largescale graph (refer to \ref{Overview of the Assembler}), thus, serving as a very good example of both data as well as compute intensive scientific workload.

Specifically, in this paper, we juxtapose the performance of different distributed cyber infrastructure with a large scale parallel genome assembler, called PGA, that we developed using Hadoop and Giraph.
We present the performance result of PGA atop three different types of clusters including LSU supercomputer, SuperMikeII and two different clusters SwatIII and CeresII located in Samsung, korea, which we built as a prototype of state-of-the-art bigdata analytics clusters.

Our performance comparison is divided into two parts.
In the first part, we provide insights on several architectural perspective including number of disk, type of storage media, amount of memory etc. in order to point out the limitations in SuperMikeII, and how we alleviate those by scaling up SwatIII and achieve almost x-speedup while using only 1/y nodes than SuperMikeII.
However, we believe that price, power and cost are as important as raw execution time for a long term deployment.
So, in the next part, we focused on comparing performance to price, space and power. 
We found almost x\% improved performance to price in SwatIII than SuperMikeII considering long term deployment.
%where we utilized the full power of each of the clusters and assembled 452GB human genome containing almost 1.5billion short reads and involves analyzing almost 4TB of graph.
%In this part our result is based upon assemblying 85GB of bumble-bee genome containing almost 500million short reads and produce almost 90GB of graph for subsequent analysis.
%Our result indicates the architetural limitations in traditional supercomputers when addressing the data intensive scientific applications and provide a good insight on the next generation high performance bigdata analytics cluster.

The rest of the paper is organized as follows:
Section-\ref{Related Work} describes the prior works related to our study.
In section-\ref{Bigdata Softwares on Traditional Supercomputers} we discuss the programming model offered by Hadoop and Giraph as well as provide a brief overview of the limitation in traditional supercomputers.
Section-\ref{Evaluation Methodology} describes the assembly workload.
In section-\ref{Result and Analysis} we evaluate different distributed cyber infrastructure for the assembly workload. 






%Optimal processing of huge amount of bigdata produced by different scientific experimental facilities poses several challneges in terms of entangled with complex scientific computation impose several challenges in terms of efficient provisioning of hardware.
%Most of the existing supercomputers being made for high performance compute intensive applications and being optimally tuned for existing HPC technologies (eg. MPI, grid etc.) failed to address these bigdata challenges.
%Although, these supercomputers provide enormous computation power in terms of tera/peta FLOPS, bigdata analysis is severely constrained by the IO and memory-availability resulting in CPU under utilization.
%However, the existing pay-as-you-go cloud infrastructures (eg. Amazon EC2, Penguin Computing, R-HPC etc.) allevietes some of these problems by providing the HPC users elasticity of resources, the performance of the existing HPC technologies on top of these cloud resources remain suboptimal.
%In the past few years several cloud resources especially Amazon EC2 is evaluated for broad range of HPC problems including several MPI and Grid benchmarks. 
%In most of the cases supercomputers with huge processing power and low latency infiniband interconnect outperformed the cloud in terms of raw performance making the supercomputers the first choice of the scientific community . 
%On the contrary, cloud is proved to be viable not only in terms of price to performance but also in eliminating the cluster setup time and cost.

%On the other hand, in the last few years the computation model shifted severely towards data driven analysis especially after the introduction of Hadoop and the rich software eco system that is built surrounding it.
%The simple abstraction provided by these massively parallel state of the art bigdata analytics softwares enable the developers to write petabyte scale applications easily.
%In the last few years we experienced several deployment of large scale Hadoop cluster in industries including Yahoo, Facebook, Samsung etc. to ease the large scale analytics rendering competative advantages.

%Not only the industries, in recent years we found a growing interest to leverage the benefit of these massively parallel system in the scientific communtiy also.
%The simple yet powerful programming model offered by these state of the art bigdata analytics softwares and their promising performance-results in different distributed cyberinfrastructucture motivates many HPC aficionados started observing several data intensive HPC challenges as a form of bigdata analytics.
%This growing interest in the HPC community creates new challenges as well as opportunities for the hardware manufaucturers in terms of efficient provisioning for these massively parallel distributed system.
%In order to address these challenges, we evaluate the performance of current state of the art bigdata softwares on different distributed cyber infrastructures with existing benchmarks as well was our parallel bigdata genome assembler based upon Hadoop and Giraph which serves as a real world example of data as well as compute intensive job.

%Although, performance of Hadoop on different types of clusters is extensively studied in the last few years [references] unfortunately most of these studies are severely limited to enterprise analytics jobs only, thereby missing the HPC aspect of bigdata entirely. 
%Furthermore, these studies are restricted in comparing any single component of a distributed system, either storage or network interconnect and unfortunately failed to address the growing demand for developing comprehensive next generation distributed cyberinfrastructure in support of bigdata high-performance scientific and engineering applications.   
%We summarize the limitations in these studies as follows,
%\begin{inparaenum}[\itshape a\upshape)]
%\item Most of the experimental work load handle no more than 100GB of input \cite{scaleupscaleout:appuswamy}\cite{scaleupscaleout:chen},
%%It is motivated by the observation, 90\% of the real world enterprise analytics jobs handle less than 100GB of input data[Microsoft][chenfb]. 
%whereas, bigdata HPC problems like genome assembly frequently deal with of data in terabytes scale. 
%\item Most of the experiments are performed on top of the existing benchmarks (eg. \cite{bm:hibench}) which are not well tested for data intensive supercomputing workload which has different computation and communication characteristics than current state of deployments of the available bigdata analytics softwares. 
%\item Graphs are a core part of many of the HPC as well as enterprise analytics workloads. 
%%Large scale graph analysis involves multiple iteration along with enormous number of random reads/writes which is traditionally addressed using shared memory approach. 
%Although Apache Giraph addressed many of the large scale graph challenges successfully, unfortunately, it's performance is hardly studied in any of the recent works.
%\item Most of the studies analyze any single component of distributed computing, either storage or network and did not see the performance as a function of the entire cluster architecture. 
%\item The metrices related to price to performance (eg. performance/\$, performance/watt etc.) are not well analyzed which leads to different conclusions regarding scaledup and scaledout deployment of Hadoop clusters. 
%Furthermore, very few of these studies consider the performance metrices for a long term deployment.
%\end{inparaenum}

%In order to address these limitations, in our work, we compare the performance of available disributed computing resources with our parallel Hadoop and Giraph based genome assembler called PGA that serves as a real world HPC problem that made its way to the forefront of bigdata challenges. 
%De novo genome assembly pipeline involves analysis of huge amount (hundreds of GB) of shortread sequences produced by high throughput genome sequencers followed by a large scale graph analysis (de Bruijn graph [pevzner]) thereby making the entire assembly problem a very good representative of Data intensive supercomputing applications workload.  
%PGA successfully assembled large scale mammalian genome including human genome of size 450GB that produces an intermidiate de Bruijn garph of size almost 4TB in sevaral distributed cyber computing infrastructure. 
%In this paper, we present the performance result of PGA on different types of cluster including LSU supercomputer QueenBee as well as two different types of clusters located in Samsung, Korea.
%[1/2 lines on Differences of the clusters in terms of storage/network]

%We present the performance result of [n] different types of Hadoop or Giraph jobs from the entire human genome assembly pipeline that handles most of the input size, there by acts as a dominating factor in deciding the amount of resources.
%We observed [any highlights/general statement on observation that we are found/interested to show],
%[eg. 
%impact of storage on network (if any)
%impact of storage on CPU.
%impact of network on CPU.]
%Our perfromance analysis shows that [any highlights on highlevel performance metrics].
%[eg.
%perfromance/\$
%performance/watt
%longterm profit]
 
%The rest of the paper is organized as follows. 
%Section 2 dicusses the related work. 
%Section 3 describes Hadoop and Giraph. 
%In section 4 we discuss the overview of our genome assembler and how much data it handles. 
%Section 4 describes the available resources. 
%Finally, an Section 5 we present and discuss our result.

\section {Related Work} \label{Related Work}
Earlier studies showed that Hadoop can be useful for data intensive scientific workloads \cite{schadoop:fadika}.
Consequently, a growing number of codes in several scientific areas such as bioinformatics, geoscience are currently being written using open source state of the art bigdata analytics software like Hadoop, Giraph etc. \cite{fw:myhadoop}.
Many of the traditional supercomputers also started using myHadoop \cite{fw:myhadoop} to provide the scientists an easy interface to configure Hadoop on-demand. 
However, there is very limited prior work that evaluated different distributed cyber infrastructures for these softwares when applied for data intensive scientific workload.
This leaves a fundamental question yet to be answered: \textit{how does a next generation  high performance computation cluster should look like to handle data intensive scientific workload}.
In this section we provide the related works for our study.

\textbf{BigData analytics softwares:}
Hadoop \cite{fw:hadoop} offers a simple, easily scalable disk-based map-reduce abstraction.
HBase \cite{fw:hbase} is a NoSQL-based distributed linearly scalable key-value store targetted the applications that need random, realtime read or write access to tera/peta byte scale data residing in disk.
Similarly, Hive \cite{fw:hive}, Impala \cite{fw:impala} etc. are some of the popular disk-based NoSQL Database which provide the users with an SQl like query interface.
On the other hand, Piccolo \cite{fw:piccolo} and Redis \cite{fw:redis} are two in-memory distributed key-value store, aimed at applications that need low-latency finegrained random access. 
Giraph \cite{fw:giraph} is a synchronous, vertex centric, in-memory graph processing framework originated as the open-source counterpart to Google's Pregel \cite{fw:pregel} that we analyzed in our work.
GraphLab \cite{fw:graphlab} is a faster asynchronous graph processing framework mainly motivated to provide the users a framework to write correct machine learning algorithms.
Resilient Distributed Datasets (RDDs) \cite{fw:rdd} in the Spark system, offers a unified in-memory solution for all batch processing, Stream prcessing \cite{fw:sparkstreaming}, SQL query \cite{fw:sparksql} and graph processing \cite{fw:graphx}.
Although, the computation model has evolved enough in the last few years to handle data intensive complex scientific workload, the choice of underlying hardware infrastructure still remains a major challenge.

\textbf{Evaluation of Hadoop for scientific workload on existing superComputers:}
With growing number of scientific applications written in Hadoop, many different groups studied the performance of Hadoop on different existing supercomputers that they have access to.
Jha \cite{schadoop:jha} observed the convergence between traditional HPC and current state of the art bigdata analytics softwares and evaluated both of them in different supercomputing environment with k-means clustering as an example.
Fadika \cite{schadoop:fadika} studied the performance of Hadoop for common HPC workload namely filter, merge and append.
Guo \cite{scgraph:guo} analyzed different graph processing framework with graph500 \cite{bm:graph500} BFS workload.
Although, thse studies provide excellent insights on performance of current state of the art bigdata analytics softwares for different scientific applications, their analysis is confined into the domain of existing supercomputers, thereby, unable to address whether or not we can get better performance in other cyber infrastructure.

\textbf{Evaluation of Hadoop for enterprise workload on different cyber infrastructure:}
Several performance analysis studies have been made with Hadoop atop different types of storages (SSD and HDD) and highspeed network interconnects (Infiniband and Ethernet etc).
Moon \cite{ssdhdd:moon} showed significant cost benefit by storing intermediate Hadoop data in SSD, leaving the HDDs to store Hadoop Distributed File System (HDFS \cite{fw:hdfs}) source data.
Wu \cite{ssdhdd:wu} found that Hadoop performance can be increased almost linearly with the increasing fraction of SSDs in the storage system.
Ahn \cite{ssdhdd:ahn} identified in a virtual environment overhead of virtualization is minimized with SSDs.
Tan \cite{ssdhdd:tan} analyzed the performance of SSD and HDD of different type of workloads involving different IO patterns and found better performance in using SSD.
Vienne \cite{ethib:vienne} evaluated the performance of Hadoop on different high speed interconnects such as 40GigE RoCE and Inifiniband FDR and found InfiniBand FDR yields the best performance for HPC as well as cloud computing applications.
Similarly, Yu \cite{ethib:yu} found improvedperformance of Hadoop in traditional supercomputers due to high speed networks.
From the perspective of a cost effective deployment, Appuswamy \cite{scaleupscaleout:appuswamy} studied the scale-out and scale-up performance for different enterprise level Hadoop job and found better perfromance price to performance in scaled up system.
On the contrary, Michael \cite{scaleupscaleout:michael} reached entirely different conclusion for interactive query.
All of the above studies have been performed either with existing benchmarks like HiBench[] or for enterprise level analytics workloads such as log processing etc, thus, unable to address the HPC aspect of Hadoop in terms of efficient hardware provisioning.
Furthermore, very limited studies consider the in-memory graph processing frameworks like Giraph, although, graph analysis is a core part of many analytics workloads.     

From the above survey, we found a gap in the existing studies in terms of evaluating different distributed cyber infrastructure for current state of the art bigdata analytics software for a real world data intensive scientific workloads.
On the other hand, millions of dollars are being invested in programs like NSFcloud with the goal "to provide an experimental platform enabling the academic research community to drive research on a new generation of innovative applications of cloud computing and cloud computing architectures" \cite{nsfcloud} . 
Hence, we found it extremely important to find out the limitations in traditional supercomputers in terms of underlying hardware infratructure and present a study addressing how to alleviate those limitations in an efficient yet cost-effective manner.
%Our study is motivated to answer the fundamental question that is being increasingly important: \textit{How does  a next generation high performance compute cluster should look like to address data intensive scientific workloads}.




%We are flooded with data today.
%Efficient processing of these data is important for modern analytics that led the development of sevaral large scale analytics frame work in last few years especially after the introduction of Hadoop, the opensource map-reduce  framework \cite{fw:mapreduce} that became the defacto standard of distributed computing.
%These massively parallel distributed computing frameworks can be broadly classified into two domains, 
%\begin{inparaenum}[\itshape a\upshape)]
%\item disk-based and  
%\item in-memory.
%\end{inparaenum}
%High volume of data and the availability of more disk space than memory in commodity hardware led the development of several disk based processing framework. 
%Whereas, with businesses and scienfic researches demanding faster access to information and requiring high performance, in-memory processing is getting more attention recently. 
%The bigdata analytics frameworks in both of these domains offer mainly three different types of programming abstractions, 
%\begin{inparaenum}[\itshape a\upshape)]
%\item Map-Reduce 
%\item NoSQL (distributed key-value store),
%\item Graph Processing.
%\end{inparaenum}

%Hadoop \cite{fw:hadoop} offers a simple, easily scalable disk-based map-reduce abstraction.
%HBase \cite{fw:hbase} is a NoSQL-based distributed linearly scalable key-value store targetted the applications that need  random, realtime read or write access to tera/peta byte scale data residing in disk.
%Similarly, Hive \cite{fw:hive}, Impala \cite{fw:impala} etc. are some of the popular disk-based NoSQL Database which provide the users with an SQl like query interface.
%Surfer \cite{fw:surfer} is a disk based map-reduce enabled graph processing framework that focused on reducing network traffic in graph processing.
%Pegasus \cite{fw:pegasus} and GBase \cite{fw:gbase} are two disk-based graph mining framework which are developed on top of Hadoop and are capable to process billion-scale graph .
%While the disk-based technologies concentrates mainly on handling large petabyte scale data, the growing need for faster computation drives the development of several in-memory data processing frameworks. 
%From that end, 'Piccolo \cite{fw:piccolo} and Redis \cite{fw:redis} are two in-memory distributed key-value store, aimed at applications that need low-latency finegrained random access to state. 
%Giraph \cite{fw:giraph} is a popular iterative, message passing based, vertex centric, scalable graph processing framework originated as the open-source counterpart to Google's Pregel \cite{fw:pregel} that we analyzed in our work.
%Giraph++ \cite{fw:giraphpp} is an improvement to giraph that reduces the number of messages.
%GraphLab \cite{fw:graphlab} is a faster asynchronous graph processing framework allowing the users a tabular as well as vertex centric view of the graph motivated to provide the users a framework to write correct machine learning algorithms.
%Resilient Distributed Datasets (RDDs) \cite{fw:rdd} in the Spark system, offers a unified in-memory solution for batch processing, SQL query and graph processing.
%X-Stream \cite{fw:xstream} and GraphChi \cite{fw:graphchi} are two disk-based, memory efficient standalone graph processing framework that enables the users to process large scale graph in a single machine. 
%However, these massively parallel distributed frameworks made the computation easier, efficient provisioning in terms of cost-effective hardwares and high performance still remain a major challenge which works as one of the motivations of our work.

%In order to show the current research trend in bigdata distributed computing and to identify the gap, we classified the ongoing researches in three major components serving as three axes in fig[].
%\begin{inparaenum}[\itshape a\upshape)]
%\item disk-based,
%\item shared-memory and
%\item Network.
%\end{inparaenum}
%All the popular diskbased petascale analytics softwares (eg. Hadoop, HBase, Surfer etc) serves the foundation of plane1. 
%Their performance depends on not only any single axes but both on storage and network architecture.
%Whereas, the performance of all the in-memory bigdata processing frameworks that lie in plane2 of fig[] (eg. GridGain, Giraph, Redis etc) depends mainly on network given enough amount of memory that can hold the data.
%We found very less research in this arena and compare performance of Giraph on different clusters with varying network architecture.
%Finally, the single-node analytics framework (eg. X-Stream, Graph Chi etc) as well as the scaledup deployment of Hadoop falls in plane3 in fig[] which rely mainly on storage architecture given enough memory that can hold the data. 

%In the last few years several performance analysis studies have been made with Hadoop atop different types of storages (SSD and HDD) and highspeed network interconnects (Infiniband and 40GigE etc).
%Moon \cite{ssdhdd:moon} showed significant cost benefit by storing intermediate Hadoop data in SSD, leaving the HDDs to store Hadoop Distributed File System (HDFS \cite{fw:hdfs}) source data.
%Wu \cite{ssdhdd:wu} found that Hadoop performance can be increased almost linearly with the increasing fraction of SSDs in the storage system.
%Ahn \cite{ssdhdd:ahn} identified in a virtual environment overhead of virtualization is minimized with SSDs.
%Tan \cite{ssdhdd:tan} analyzed the performance of SSD and HDD of different type of workloads involving different IO patterns and found better performance in using SSD.
%Vienne \cite{ethib:vienne} evaluated the performance of Hadoop on different high speed interconnects such as 40GigE RoCE and Inifiniband FDR and found InfiniBand FDR yields the best performance for HPC as well as cloud computing applications.
%Similarly, Yu \cite{ethib:yu} found improvedperformance of Hadoop in traditional supercomputers due to high speed networks
%Although, these studies unanimously showed the benefit of SSD over HDD, their experiments were performed on existing benchmarks (eg. intel hibench \cite{bm:hibench}) which are not well tested for real world data intensive high performance computing workload. We address this challenge using our Hadoop and Giraph based bigdata genome assembler that serves as a very good example of both data as well as compute intensive jobs.
%Also, these studies being restricted in comparing only a single component of the cluster failed to provide an overview of next generation clusters. 

%From the perspective of a cost effective deployment, Appuswamy \cite{scaleupscaleout:appuswamy} studied the scale-out and scale-up performance for different enterprise level Hadoop job and found better perfromance price to performance in scaled up system.
%On the contrary, Michael \cite{scaleupscaleout:michael} reached entirely different conclusion for interactive query.
%These contradictory results in two different workloads in two different types of cluster deployment drive us to rethink about the existing analysis procedure of different price to performance metrices in the context of longterm HPC cluster deployment.
%Furthermore, most of these scaledup and scaledout comparison is driven by the observation that 90\% of the enterprise level jobs handle no more than 100GB of data \cite{scaleupscaleout:chen}, thus can be accomodated in a single server with high memory and disk space. 
%This assumption hardly holds for a BigData HPC cluster deployment.


%\section {Bigdata software and HPC}
%\subsection {Hadoop}
Hadoop was originated as the opensource counter part of Google's Map-Reduce \cite{fw:mapreduce}.
Hadoop has two different components: Hadoop Distributed File System (HDFS) and a mapreduce programming abstarction.
HDFS splits huge volume of data into small disjoint sets called blocks (typically of size 64mb to 128mb) and distributes those accross the cluster.
A user defined map function is applied to each blocks parallely in order to extract information from each records in the form of key-value pair.
These intermidiate key-value pairs are then partitioned on the basis of keys where each key gets a list of values.
Finally, a user defined reduce function is applied to the value-list of each key independently and the final output is wrtten to HDFS.

%In the last decade the introcuction of Hadoop, the open-source implementation of Google's map-reduce revolutionized our ability to find signals in noise.
%Hadoop-mapreduce coupled with Hadoop Distributed File System (HDFS) provides solution for a storage and computation of large scale data.
%Hadoop-mapreduce consists of three different phases. 
%\begin{inparaenum}[\itshape a\upshape)]
%\item Map
%\item shuffle
%\item reduce
%\end{inparaenum}
%In the first phase the input data kept in HDFS is splitted into several disjoint sets of records called blocks and a user defined map function is applied to each split independently without interacting each other in order to extract information from each record in the form of key, value pair.
%In the second phase these intermidiate key, value pairs are partitioned on the basis of the keys where each key gets a list of values and then it is transferred to the reducers.
%Finally, a user defined reduce function is applied to the value-list of each key independently and the final output is wrtten to HDFS.
%The shared nothing architecture in both map and reduce phase classifies Hadoop as Single Instruction Multiple data (SIMD) in Flynn's taxonomy and enable it to exploit data level parallelism with no side effect.
%Besides this, the computation in Hadoop is tuned to exploit the data locality.  These characteristics of Hadoop fit a broad range of existing high performance scientific computation workflow and gain enough popularity as a potential alternative to the decade old HPC technologies like MPI or Grid.

%Despite of this potential, the bigdata in tera or peta byte scale and limitation of memory in traditional supercomputers make both the map and reduce process of Hadoop disk-io bound limiting the HPC users and developers to exploit the power of CPU.
%Furthermore, the shuffle phase involves huge amount of data flow over network which makes the reduce process network bound also.
%These limitations create challenges as well as opportunities for the hardware manufacturers both in the domains of storage and networking in terms of efficient hardware provisioning which yield not only raw performance benefit but also better performance to price.
  
\subsection {Giraph}
Large scale graph analysis is a core part of many supercomputing workload.
%Graph processing inherently involves many iteration and random access to the data and trditionally addressed with in-memory technologies.
Apache Giraph is an iterative in-memory grpah processing framework that is implemented on top of Hadoop's map-reduce implementation.
It is originated as the open-source counterpart to Google's Pregel \cite{fw:pregel}.
Giraph is inspired by Bulk Synchronous Parallel model \cite{fw:bsp} where computation proceeds in supersteps.
In each superstep all vertices of the graph excutes different istances of the same progam called vetrex-program simultaneosly without ineracting with other verties which is similar to map tasks of Hadoop
After each superstep all the vertices send messages to other vertices normally cotaining the output of its vertex-progam-instance.
Once all the messages are recieved by the intended vertices, the next supesrtep starts and the pocess iterates until all the vertices vote to halt simultaeously.


%Large scale graph analysis is a core part of many data intensive supercomputing workload which involves many iteration and random access to the data.
%On the other hand, the performance of Hadoop is severely constrained by its iterative limitation and HDFS in its native form is optimized for sequential access only, which make Hadoop extremely inefficient for large scale graph processing.
%Giraph, the open surce implmentation of Googles Pregel is a popular graph processing framework that is developed on top of Hadoop and finds its base in Bulk Synchronus processing. 
%Giraph provides its users a shared-memory based vertex centric, synchronous grah processing model which relies on mesage passing communicaton.
%The computation in Giraph is divided into a series of supersteps.
%In each superstep all vertices of the graph excutes different istances of the same progam called vetrex-program simultaneosly without ineracting with other verties which is similar to map tasks of Hadoop
%After each superstep all the vertices send message normally cotaining the output of its vertex-progam-instance to others.
%Once all the messages are recieved by the intended vertices, the next supesrtep starts and the pocess iterates until all the vertices vote to halt simultaeously.
%Like Hadoop map phase, Giraph can also be classified as SIMD due to its vertex centric prallelism, however, there is a message passing enabled between map tasks.
%This in-memory data parllelism with messge passing makes Giraph a good choice for many HPC problem.

%Unforunately, the performance characterstics of this model either in terms of memory, CPU or network for large volume of data is hardly studied which crates a gap in studies addessing the growing demand for developing comprehensive next generation distributed cyberinfrastructure in support of bigdata high-performance scientific and engineering applications. 


%\section {Workload}
%De novo genome assembly refers to construction of an entire genome sequence from a large amount of short read sequences when no reference genome is available. 
De Bruijn graph construction  and removal of sequencing errors (tips and bubble) from this graph is central to de novo sequencing. 
Finally, resolving  repeated regions followed by a scaffolding phase produces long size scaffolds that represents a region in the actual genome.

We classified de novo sequencing in three different phases like other assemblers.
\begin{inparaenum}[\itshape a\upshape)]
\item De Bruijn graph construction
\item Graph simplification and
\item Scaffolding.
\end{inparaenum}
We store short reads in fastq format in hdfs as input to PGA.
In the first phase, we use Hadoop in order to build de Bruijn graph from these short reads. 
Once the graph is constructed we use Giraph in the subsequent phases to analyze the graph in order to construct appreciably long contigs and scaffolds.
In this section we provide a brief overview of each stage of the assembler.

\subsubsection {De Bruijn graph construction}
This phase is inspired by Contrail, another Hadoop based genome assembler.
The de Brujin graph is constructed using MapReduce by scanning each read in the mapper.
In the map phase, each read is divided into several short fragments of length k known as k-mer.
Each two subsequent k-mers are emitted as key-value pairs where the first one (key) represents a vertex in the graph and the second one (value) represents the outgoing edge from the key.
Similar process is repeated for the reverse complement of the reads are also considered.
After the map function completes, the shuffle phase partitions the intermidiate key-value pairs on the basis of key which effectively collects the edges of the graph emitted from the same source k-mer.
Finally, the reduce function aggregates the edges (value-list) of each source k-mer and saves the graph structure in HDFS.

\subsubsection {Graph Simplification}
This phase invokes a series of Giraph jobs each of which perform the following and write the output to the HDFS which seves as the input to the next Giraph job.
The process continues until no linear chains are found in the graph. 
 
\textbf{Compression:} The first step that follows after building the graph is compressing the linear chain of nodes in the graph.
Giraph reads the graph from HDFS in a predefined adjacency list format where the source k-mer serves both as vertex id and value.
The non-branching linear paths of vertices are compressed into single vertex without the loss of any information.
In one superstep each compressible vertex is tagged as either head or tail with equal probability and send a meassage containing the tag to the immidiate predecessor and successor.
In the next superstep the head-vertices are merged with corresponding tail-vertices.
The value of the head is updated accordingly. 
This process continues for $i$ supersteps until there is no compressible vertex remaining in the graph.

\textbf{Tip removal:} Tips are formed because of errors in the end of the short reads.
Removing the tips from the de Bruijn Graph is a straight forward process.
After compressing the graph, in a single superstep the vertices with no outgoing edge and the value-length less than a threshold (normally set to $2k$) are deleted from the graph.

\textbf{Bubble removal:} Bubbles are introduced in the DBG because of errors in the middle of the short reads.
Bubbles are formed when two paths start and end at the same vertices.
After compressing the linear chain, the objective of bubble removal is to group the vertices by the same predecessor and successor in the entire graph and from each group keep only the node which has the highest frequency support.
In one superstep every node matching this criteria sends the cumulative frequency to their immidiate successor.
In the next superstep successor nodes compute difference in frequency and delete all the nodes with lower frequency.
Remember we calculated the frequency during the compression phase.

\subsubsection {Scaffolding}
The first step of scaffolding determines which contigs are linked by matepairs, and their relative orientation and separation. By convention, mated reads have the same name except for their suffix (either 1 or 2). 
PGA therefore finds all mate-linked contigs using a single MapReduce cycle by emitting from the mapper mate messages consisting of the read name without the suffix as the key, and the contig name, read orientation, and read offset as the value.

Next, we developed a graph hop method to find the exact path between the linked nodes

\section {Bigdata Softwares on Traditional Supercomputers} \label{Bigdata Softwares on Traditional Supercomputers}
\subsection {Hadoop}
Hadoop was originated as the opensource counter part of Google's Map-Reduce \cite{fw:mapreduce}.
Hadoop has two different components: Hadoop Distributed File System (HDFS) and a mapreduce programming abstarction.
HDFS splits huge volume of data into small disjoint sets called blocks (typically of size 64mb to 128mb) and distributes those accross the cluster.
A user defined map function is applied to each blocks parallely in order to extract information from each records in the form of key-value pair.
These intermidiate key-value pairs are then partitioned on the basis of keys where each key gets a list of values.
Finally, a user defined reduce function is applied to the value-list of each key independently and the final output is wrtten to HDFS.

%In the last decade the introcuction of Hadoop, the open-source implementation of Google's map-reduce revolutionized our ability to find signals in noise.
%Hadoop-mapreduce coupled with Hadoop Distributed File System (HDFS) provides solution for a storage and computation of large scale data.
%Hadoop-mapreduce consists of three different phases. 
%\begin{inparaenum}[\itshape a\upshape)]
%\item Map
%\item shuffle
%\item reduce
%\end{inparaenum}
%In the first phase the input data kept in HDFS is splitted into several disjoint sets of records called blocks and a user defined map function is applied to each split independently without interacting each other in order to extract information from each record in the form of key, value pair.
%In the second phase these intermidiate key, value pairs are partitioned on the basis of the keys where each key gets a list of values and then it is transferred to the reducers.
%Finally, a user defined reduce function is applied to the value-list of each key independently and the final output is wrtten to HDFS.
%The shared nothing architecture in both map and reduce phase classifies Hadoop as Single Instruction Multiple data (SIMD) in Flynn's taxonomy and enable it to exploit data level parallelism with no side effect.
%Besides this, the computation in Hadoop is tuned to exploit the data locality.  These characteristics of Hadoop fit a broad range of existing high performance scientific computation workflow and gain enough popularity as a potential alternative to the decade old HPC technologies like MPI or Grid.

%Despite of this potential, the bigdata in tera or peta byte scale and limitation of memory in traditional supercomputers make both the map and reduce process of Hadoop disk-io bound limiting the HPC users and developers to exploit the power of CPU.
%Furthermore, the shuffle phase involves huge amount of data flow over network which makes the reduce process network bound also.
%These limitations create challenges as well as opportunities for the hardware manufacturers both in the domains of storage and networking in terms of efficient hardware provisioning which yield not only raw performance benefit but also better performance to price.
\subsection {Giraph}
Large scale graph analysis is a core part of many supercomputing workload.
%Graph processing inherently involves many iteration and random access to the data and trditionally addressed with in-memory technologies.
Apache Giraph is an iterative in-memory grpah processing framework that is implemented on top of Hadoop's map-reduce implementation.
It is originated as the open-source counterpart to Google's Pregel \cite{fw:pregel}.
Giraph is inspired by Bulk Synchronous Parallel model \cite{fw:bsp} where computation proceeds in supersteps.
In each superstep all vertices of the graph excutes different istances of the same progam called vetrex-program simultaneosly without ineracting with other verties which is similar to map tasks of Hadoop
After each superstep all the vertices send messages to other vertices normally cotaining the output of its vertex-progam-instance.
Once all the messages are recieved by the intended vertices, the next supesrtep starts and the pocess iterates until all the vertices vote to halt simultaeously.


%Large scale graph analysis is a core part of many data intensive supercomputing workload which involves many iteration and random access to the data.
%On the other hand, the performance of Hadoop is severely constrained by its iterative limitation and HDFS in its native form is optimized for sequential access only, which make Hadoop extremely inefficient for large scale graph processing.
%Giraph, the open surce implmentation of Googles Pregel is a popular graph processing framework that is developed on top of Hadoop and finds its base in Bulk Synchronus processing. 
%Giraph provides its users a shared-memory based vertex centric, synchronous grah processing model which relies on mesage passing communicaton.
%The computation in Giraph is divided into a series of supersteps.
%In each superstep all vertices of the graph excutes different istances of the same progam called vetrex-program simultaneosly without ineracting with other verties which is similar to map tasks of Hadoop
%After each superstep all the vertices send message normally cotaining the output of its vertex-progam-instance to others.
%Once all the messages are recieved by the intended vertices, the next supesrtep starts and the pocess iterates until all the vertices vote to halt simultaeously.
%Like Hadoop map phase, Giraph can also be classified as SIMD due to its vertex centric prallelism, however, there is a message passing enabled between map tasks.
%This in-memory data parllelism with messge passing makes Giraph a good choice for many HPC problem.

%Unforunately, the performance characterstics of this model either in terms of memory, CPU or network for large volume of data is hardly studied which crates a gap in studies addessing the growing demand for developing comprehensive next generation distributed cyberinfrastructure in support of bigdata high-performance scientific and engineering applications. 
\subsection {Limitations in traditional supercomputers}
Earlier studies \cite{schadoop:fadika}, \cite{schadoop:matsunaga} as well as our experience show that Hadoop and other softwares in its eco system like Giraph can be useful for data-intensive scientific applications, however, the underlying storage, memory as well as the computation model differs severely from other parallel processing frameworks like MPI. 
The challenges involved in optimal processing of these data-intensive workload needs to be addressed possibly by changing the underlying hardware infrastructure.
In this section we provide a brief overview on the limitations in existing supercomputers.

\textbf{Storage:} 
In order to provide high io-bandwidth, Hadoop colocates data and computation. 
Unlike other parallel file system like Lustre which stores data in dedicated io-servers, HDFS relies on local file system.
It stores the data in the same nodes where the computation takes place requiring high storage space in the compute nodes. 
Furthermore, the intermidiate output of each mapper is temporarily stored in the local filesystem of the corresponding node, which may be a magnitude higher than the final output especially in the case of a shuffle intensive job.
On the other hand, in a traditional supercomputing environment each compute node is provided with less amount of storage typically provided with one disk ranging from 250gb to 500gb.
This small amount of storage not only limits data size to be handled but also slow down the process because of lower IOPS.
Although, scaling out in terms of compute nodes may alleviate both of these issues, it does in the cost of lower CPU utilization.
In the subsequent sections, we show how number of disk per node, as well as the type of the storage media (SSD/HDD) impact the performance and price of a Hadoop workload in the context of large scale genome assembly. 

\textbf{Memory:}
Graph processing typically involves many iteration and random access to the data which is conventionally addressed with in-memory solutions. 
%Consequently their performance is severely limited by the lower memory-speed (compare to higher CPU-speed) commonly known as memory-wall.
Memory system that is used in most of the supercomputers shows lower capacity per core and fewer independent channels \cite{bm:graph500}.
Complicating the scenario, the performance is again hindered by high message passing over network for Giraph, which is designed to facilitate BSP model.
In our study, we show the impact of provisioning more memory per core in a Giraph workload.

\textbf{Swapping/Paging:}
Programs handling huge amount of data frequently perform swap in and out between memory and swap-space of the disk especially when the memory per core is small.
Traditional supercomputers with small amount of memory per core and small amount of storage per compute node frequently run out of memory as well as swap-space, consequently failing the entire job.
Furthermore, normal HDD with less throughput (than SSD) spends lots of time for io in case of swapping which adversely affects the performance of the application.
In our analysis, we tweak the in-memory java-heap-space and the swap-space (both both SSD and HDD) in our prototype bigdata analytics cluster to understatnd its impact on the performance as well as the price.


\section {Evaluation Methodology} \label{Evaluation Methodology}
%Although, Hadoop and the related softwares in its ecosystem were initially developed to support the cheap data storage and enterprise level analytics workloads, a convergence with HPC at many different levels has been found especially in terms of programming model.
%%Earlier studies[][][][] as well as our experience show that the programming model offered by these softwares can support many different HPC workloads.  
%As a consequence, the traditional approach to deploy Hadoop atop scaled out cluster of commodity hardwares has been changed at least in the context of HPC.

%On the other hand, unlike traditional HPC technologies like MPI, Grid etc Hadoop colocates data and computation in order to support flat scalability.
%Furthermore, lack of POSIX support, depency on Java, etc. make the Hadoop workloads fundamentally different from the traditional high performance technologies.
%Complicating the scenario, the cumulative effect of io, network and memory bandwidth on the overall performance of the data intensive workload make the process of providing cost-effective and efficient hardware extremely difficult. 
%There is very little understanding of the performance tradeoffs of different storage or network interconnects when Hadoop and Giraph is applied for data intensive high performance scientific applications like large scale genome assembly. 
%In this section we provide the overview of our evaluation methodology.

%In this section we describe the input data size follwed by the descripion of experimental testbeds that we used for our prallel genome assembly application.
%Then we compare the CPU and network characteristics of Intel hiBench, an existing benchmark suite developed to evaluate Hadoop performance and the first mapreduce phase of PGA that is the de Bruijn graph construction from shortreads.
%Then, we present the cluster characteristics in terms of CPU, network and storage for both the Hadoop and Giraph phase of PGA separately.
%Finally, we analyze the performance result of PGA and show that [one of the clusters] yields better execution time whereas, [the other] yields better price to performance.
\subsection {Genome Assembly Workload}
De novo genome assembly refers to construction of an entire genome sequence from a large amount of short read sequences when no reference genome is available. 
De Bruijn graph construction  and removal of sequencing errors (tips and bubble) from this graph is central to de novo sequencing. 
Finally, resolving  repeated regions followed by a scaffolding phase produces long size scaffolds that represents a region in the actual genome.

We classified de novo sequencing in three different phases like other assemblers.
\begin{inparaenum}[\itshape a\upshape)]
\item De Bruijn graph construction
\item Graph simplification and
\item Scaffolding.
\end{inparaenum}
We store short reads in fastq format in hdfs as input to PGA.
In the first phase, we use Hadoop in order to build de Bruijn graph from these short reads. 
Once the graph is constructed we use Giraph in the subsequent phases to analyze the graph in order to construct appreciably long contigs and scaffolds.
In this section we provide a brief overview of each stage of the assembler.

\subsubsection {De Bruijn graph construction}
This phase is inspired by Contrail, another Hadoop based genome assembler.
The de Brujin graph is constructed using MapReduce by scanning each read in the mapper.
In the map phase, each read is divided into several short fragments of length k known as k-mer.
Each two subsequent k-mers are emitted as key-value pairs where the first one (key) represents a vertex in the graph and the second one (value) represents the outgoing edge from the key.
Similar process is repeated for the reverse complement of the reads are also considered.
After the map function completes, the shuffle phase partitions the intermidiate key-value pairs on the basis of key which effectively collects the edges of the graph emitted from the same source k-mer.
Finally, the reduce function aggregates the edges (value-list) of each source k-mer and saves the graph structure in HDFS.

\subsubsection {Graph Simplification}
This phase invokes a series of Giraph jobs each of which perform the following and write the output to the HDFS which seves as the input to the next Giraph job.
The process continues until no linear chains are found in the graph. 
 
\textbf{Compression:} The first step that follows after building the graph is compressing the linear chain of nodes in the graph.
Giraph reads the graph from HDFS in a predefined adjacency list format where the source k-mer serves both as vertex id and value.
The non-branching linear paths of vertices are compressed into single vertex without the loss of any information.
In one superstep each compressible vertex is tagged as either head or tail with equal probability and send a meassage containing the tag to the immidiate predecessor and successor.
In the next superstep the head-vertices are merged with corresponding tail-vertices.
The value of the head is updated accordingly. 
This process continues for $i$ supersteps until there is no compressible vertex remaining in the graph.

\textbf{Tip removal:} Tips are formed because of errors in the end of the short reads.
Removing the tips from the de Bruijn Graph is a straight forward process.
After compressing the graph, in a single superstep the vertices with no outgoing edge and the value-length less than a threshold (normally set to $2k$) are deleted from the graph.

\textbf{Bubble removal:} Bubbles are introduced in the DBG because of errors in the middle of the short reads.
Bubbles are formed when two paths start and end at the same vertices.
After compressing the linear chain, the objective of bubble removal is to group the vertices by the same predecessor and successor in the entire graph and from each group keep only the node which has the highest frequency support.
In one superstep every node matching this criteria sends the cumulative frequency to their immidiate successor.
In the next superstep successor nodes compute difference in frequency and delete all the nodes with lower frequency.
Remember we calculated the frequency during the compression phase.

\subsubsection {Scaffolding}
The first step of scaffolding determines which contigs are linked by matepairs, and their relative orientation and separation. By convention, mated reads have the same name except for their suffix (either 1 or 2). 
PGA therefore finds all mate-linked contigs using a single MapReduce cycle by emitting from the mapper mate messages consisting of the read name without the suffix as the key, and the contig name, read orientation, and read offset as the value.

Next, we developed a graph hop method to find the exact path between the linked nodes
\subsection {Input Data}
High throughput next generation DNA sequencing machines like Illumina Genome Analyzer produce huge amount of short read sequences typically in the scale of several GigaBytes to Terabytes.
Furthermore, the size of the de Bruijn graph built from these vast amount of short reads may be another magnitude higher than the reads itself making the entire assembly pipe line severely data-intensive.
In this paper, we use the bumble bee genome and human genome sequence as a representative data sets.
The first one, bumble bee genome is available in Genome Assembly Gold-standard Evaluation (GAGE \cite{bio:gage}) website in fastq format.
The data size is 85GB containing almost 1billion reads.
The size of the de Bruijn graph produced by it is 90GB.
The second data set, the human genome is openly avalible in NCBI web site with accession no. SRX10639 in a compressed SRA format.
We decompress it using NCBI SRA toolkit which produce 452GB of short read sequence data in fastq format.
The size of the de Bruijn graph produced by it is almost 3.8TB which is almost 9 times higher than the reads itself.
The variation between the size of the input reads and the corresponding de Bruijn graph depends upon the sequencing-quality or pre-assembly-processing \cite{bio:quality1}, \cite{bio:quality2} that is out of the scope of this paper. 


%The size of the de bruijn graph varies severely with the quality of the sequencing experiments which 


%Fig. 1a shows the size of intermediate shuffled data and the actual size of the de Bruijn graph generated by the graph building phase of PGA for some of the data sets. For large scale genome assembly (eg. human genome) the shuffled data indicates very high local disk space requirment, whereas, the actual graph size indicates the huge memory requirement for in-memory graph analysis with Giraph in subsequent phases.
%In this paper we present the performance result  of different clusters with PGA for the human genome assembly.
%The data set is openly available in NCBI website with accession no. SRX10639 in a compressed SRA format.
%We decompress it  using NCBI SRA toolkit which produce 452GB of short read sequence data in fastq format which is stored in HDFS and serves as the input to PGA.
%In the first stage of PGA which is a Hdoop job to construct de Bruin graph, we create a 3.8TB of graph from these short read sequences which is anayzed in memory with Giraph in the succesive phases.
%Furthermore,the entire process of human genome assemby produces almost 15TB of temporary data that need to be stored in HDFS.
%In the next subsection we describe the resources that we used to accomodate and analyze this huge volume of data 
\subsection {Experimental Testbeds}
We compare the result of three different clusters with varying architecture.
In this section we prvide the details of the clusters.
The first one is SuperMikeII, the LSU HPC resource that represents the domain of traditional supercomputer.
This LSU supercomputer offers total 382 computing nodes, however, maximum of 128 can be allocated at a time. 
Each node has two 2.6GHz 8-core Sandy Bridge Xeon 64-bit processors, 32GB RAM, thus offering maximum 2GB of memory per core. 
Each node has 500GB local storage (normal Hard Disk Drive) available as filesystem data storage for Hadoop.
All nodes are connected through 40Gb/s infiniband network with 2:1 blocking ratio.

On the other hand, we built a prototype of the current state of the art bigdata analytics cluster, SwatIII in Samsung, Korea with HDD and SSD variant to observe the impact of different storage media on Hadoop and Giraph job.
Each node of SwatIII is equipped with sixteen 2-core Intel Xeon 64bit processors, 256GB RAM, thus yielding maximum 8GB memory per core.
Each node in this cluster has 1 TB of local storage in either variant.
Observe, each node in SwatIII offers almost double processing power than each node in SuperMikeII in terms of number of cores. 
However, in all the experiments discussed in this paper we use the same number of total cores in both the clusters in order to eliminate the impact of procesing power and do a fair comparison in terms of the effect of io-bandwidth, network-bandwith and memory-bandwidth/core which is the main focus of this paper.

\begin{center}
\begin{table}
    \begin{tabular}{ |p{1.3cm} | p{1.3cm} | p{1.3cm} | p{1.3cm} | p{1.3cm} |} \hline
    & Super MikeII @LSU & Big CRON @LSU & SwatIII @Samsung & CeresII @Samsung \\ \hline
    Number of nodes & 128 & 128 & 15 & 33 \\ \hline
    Processor & 2*2.6GHz 8-Core Sandy Bridge Xeon 64-bit & 2*2.4GHz 4-Core Intel Xeon 64-Bit & 16*2.6GHz 2-core Xeon 64-bit & 2.3GHz 2-core Xeon 64-bit \\ \hline \hline 
    Number of cores per node & 16 & 8 & 32 & 2 \\ \hline
    Total number of cores & 2048 & 1024 & 512 & 66 \\ \hline \hline
    RAM per node (GB) & 32 & 32 & 256 & 16 \\ \hline
    Total memory & 4TB & 4TB & 4TB & 528GB \\ \hline \hline
    Storage per node & 500GB HDD & 1TB HDD & 1TB HDD/ SSD & 256GB HDD \\ \hline
    Total Storage & 64TB HDD & 64TB HDD & 16TB HDD/ SSD & 7TB HDD \\ \hline \hline
    Network & 40Gb/s Inifiniband 2:1 bloking & 40Gb/s Infiniand no blocking & 10Gb/s Ethernet & 10Gb/s virtualized Ethernet\\ \hline
    \end{tabular}
    \caption{Tetbeds}
	\label{table:testbeds}
\end{table}
\end{center}

\section {Result and Analysis} \label{Result and Analysis}
In this section we present the performance result of different clusters to understand the impact of different storage and network architechitecture.
All experiments are performed as many times until 95\% confidence level is reached.

\subsection {Understanding the workload}
Figure-\ref{figResourceUsage} shows the average disk and memory utilization of PGA for assemblying bumble bee and human genome for three different phases of PGA.
The first phase, building de Bruijn graph is a shuffle intensive mapreduce job which writes almost xGB intermidiate data for bumble bee and almost xGB for human genome however the final output of the reducer that is written to HDFS is 85GB and 3.8TB respectively for bumble bee and human genome.
In the second phase, the entire graph is taken in the memory and analyzed using Giraph making it severely memory intensive.
At the same time, the output of each giraph job is written to HDFS which serves as the input to the next job making the perfromance of this phase disk-bound also.

\subsection {Measurement baseline}
In order to do both a fair comparison as well as pinpointing the limitations we changed SwatIII environent according to SuperMikeII specifications.
The parameters that we changed are listed in table-\ref{table}.
Figure-\ref{figBaseline} shows performance of assemblying bumble bee genome in using 256 cores, 16 disks and an aggregated 1TB of RAM space.
Then in the subsequent sections we turn on each feature one by one and compare its benefit in order to provide a guideline which needs to be taken care of. 

\subsection {Effect of number of disk per node}.
In this experiment we added more(??) number of storage drive in the SWATIII cluster and distributed the intermidiate output of Hadoop according to that.
Since the total IOPS increases in the cluster Hadoop was expected to yield better performance.
Figure-\ref{figIops} shows the result of adding more number of storage drive in the cluster.
Also it is important to notice that, adding each SSD to the cluster shows almost ??\% benefit over adding HDD.

\subsection {Effect of Hyper threading}
Figure-\ref{figHyperthread} shows the execution time of each phase of PGA when we enable hyperthreading in SWATIII.
In each variant of SWATIII we noticed almost ?? speedup simply enabling the hyperthreading.
On the other hand, we observed almost ?? speedup when normalized the performance with SuperMikeII.

\subsection {Effect of Java Heap Space and Virtual Memory}
The impact of java heap space and virtual memory is a long standing issue.
Although, the performance is found to vary according to the java heap space allocated to each worker the proper explanation is still unknown.
In this section we tweak the java heap space and the Linux virtual memory configuration to notice the effect of swapping on different types of storage media.
Figure-\ref{figJvm} shows that SSD with more IO throughput performs better than the HDD in SWATIII with lower amount of Virtual H.
Also, 

\subsection {Effect of adding more memory per node}

\subsubsection {Discussion}




\subsection {Performance to price comparison}

\section {Conclusion}
%The conclusion goes here.
%Despite of the fundamental differences in computation and communication characteristics involved in these two paradigms, the promising performance result of these state of the art bigdata analytics software on different distributed cyber infrastructure.



% conference papers do not normally have an appendix


% use section* for acknowledgement
\section*{Acknowledgment}


The authors would like to thank...





% trigger a \newpage just before the given reference
% number - used to balance the columns on the last page
% adjust value as needed - may need to be readjusted if
% the document is modified later
%\IEEEtriggeratref{8}
% The "triggered" command can be changed if desired:
%\IEEEtriggercmd{\enlargethispage{-5in}}

% references section

% can use a bibliography generated by BibTeX as a .bbl file
% BibTeX documentation can be easily obtained at:
% http://www.ctan.org/tex-archive/biblio/bibtex/contrib/doc/
% The IEEEtran BibTeX style support page is at:
% http://www.michaelshell.org/tex/ieeetran/bibtex/
%\bibliographystyle{IEEEtran}
% argument is your BibTeX string definitions and bibliography database(s)
%\bibliography{IEEEabrv,../bib/paper}
%
% <OR> manually copy in the resultant .bbl file
% set second argument of \begin to the number of references
% (used to reserve space for the reference number labels box)
\begin{thebibliography}{1}
\bibitem{fw:mapreduce}
Dean, Jeffrey, and Sanjay Ghemawat. "MapReduce: simplified data processing on large clusters." Communications of the ACM 51, no. 1 (2008): 107-113.
\bibitem{fw:pregel}
Malewicz, Grzegorz, Matthew H. Austern, Aart JC Bik, James C. Dehnert, Ilan Horn, Naty Leiser, and Grzegorz Czajkowski. "Pregel: a system for large-scale graph processing." In Proceedings of the 2010 ACM SIGMOD International Conference on Management of data, pp. 135-146. ACM, 2010.
\bibitem{fw:bsp}
Cheatham, Thomas, Amr Fahmy, Dan Stefanescu, and Leslie Valiant. "Bulk synchronous parallel computing—a paradigm for transportable software." In Tools and Environments for Parallel and Distributed Systems, pp. 61-76. Springer US, 1996.
\bibitem{fw:hadoop}
White, Tom. Hadoop: the definitive guide: the definitive guide. " O'Reilly Media, Inc.", 2009.
\bibitem{fw:hdfs}
Borthakur, Dhruba. "The hadoop distributed file system: Architecture and design." Hadoop Project Website 11, no. 2007 (2007): 21.
\bibitem{fw:hbase}
George, Lars. HBase: the definitive guide. " O'Reilly Media, Inc.", 2011.
\bibitem{fw:hive}
Thusoo, Ashish, Joydeep Sen Sarma, Namit Jain, Zheng Shao, Prasad Chakka, Suresh Anthony, Hao Liu, Pete Wyckoff, and Raghotham Murthy. "Hive: a warehousing solution over a map-reduce framework." Proceedings of the VLDB Endowment 2, no. 2 (2009): 1626-1629.
\bibitem{fw:impala}
Wanderman-Milne, Skye, and Nong Li. "Runtime Code Generation in Cloudera Impala." IEEE Data Eng. Bull. 37, no. 1 (2014): 31-37.
%\bibitem{fw:tajo}

\bibitem{fw:surfer}
Chen, Rishan, Xuetian Weng, Bingsheng He, Mao Yang, Byron Choi, and Xiaoming Li. "On the efficiency and programmability of large graph processing in the cloud." Microsoft Research TechReport (2010).
\bibitem{fw:pegasus}
Kang, U., Charalampos E. Tsourakakis, and Christos Faloutsos. "Pegasus: A peta-scale graph mining system implementation and observations." In Data Mining, 2009. ICDM'09. Ninth IEEE International Conference on, pp. 229-238. IEEE, 2009.
\bibitem{fw:gbase}
Kang, U., Hanghang Tong, Jimeng Sun, Ching-Yung Lin, and Christos Faloutsos. "Gbase: a scalable and general graph management system." In Proceedings of the 17th ACM SIGKDD international conference on Knowledge discovery and data mining, pp. 1091-1099. ACM, 2011.
%\bibitem{fw:GridGain}

\bibitem{fw:rdd}
Zaharia, Matei, Mosharaf Chowdhury, Tathagata Das, Ankur Dave, Justin Ma, Murphy McCauley, Michael J. Franklin, Scott Shenker, and Ion Stoica. "Resilient distributed datasets: A fault-tolerant abstraction for in-memory cluster computing." In Proceedings of the 9th USENIX conference on Networked Systems Design and Implementation, pp. 2-2. USENIX Association, 2012.
\bibitem{fw:sparkstreaming}
Zaharia, Matei, Tathagata Das, Haoyuan Li, Scott Shenker, and Ion Stoica. "Discretized streams: an efficient and fault-tolerant model for stream processing on large clusters." In Proceedings of the 4th USENIX conference on Hot Topics in Cloud Ccomputing, pp. 10-10. USENIX Association, 2012.
\bibitem{fw:graphx}
Gonzalez, Joseph E., Reynold S. Xin, Ankur Dave, Daniel Crankshaw, Michael J. Franklin, and Ion Stoica. "Graphx: Graph processing in a distributed dataflow framework." In Proceedings of the 11th USENIX Symposium on Operating Systems Design and Implementation (OSDI). 2014.
\bibitem{fw:sparksql}
Xin, Reynold S., Josh Rosen, Matei Zaharia, Michael J. Franklin, Scott Shenker, and Ion Stoica. "Shark: SQL and rich analytics at scale." In Proceedings of the 2013 ACM SIGMOD International Conference on Management of data, pp. 13-24. ACM, 2013.
\bibitem{fw:redis}
Carlson, Josiah L. Redis in Action. Manning Publications Co., 2013.
\bibitem{fw:piccolo}
Power, Russell, and Jinyang Li. "Piccolo: Building Fast, Distributed Programs with Partitioned Tables." In OSDI, vol. 10, pp. 1-14. 2010.
\bibitem{fw:giraph}
Avery, Ching. "Giraph: Large-scale graph processing infrastructure on hadoop." Proceedings of the Hadoop Summit. Santa Clara (2011).
\bibitem{fw:giraphpp}
Tian, Yuanyuan, Andrey Balmin, Severin Andreas Corsten, Shirish Tatikonda, and John McPherson. "From" think like a vertex" to" think like a graph." Proceedings of the VLDB Endowment 7, no. 3 (2013): 193-204.
\bibitem{fw:graphlab}
Low, Yucheng, Joseph E. Gonzalez, Aapo Kyrola, Danny Bickson, Carlos E. Guestrin, and Joseph Hellerstein. "Graphlab: A new framework for parallel machine learning." arXiv preprint arXiv:1408.2041 (2014).
\bibitem{fw:graphx}
Xin, Reynold S., Joseph E. Gonzalez, Michael J. Franklin, and Ion Stoica. "Graphx: A resilient distributed graph system on spark." In First International Workshop on Graph Data Management Experiences and Systems, p. 2. ACM, 2013.
\bibitem{fw:xstream}
Roy, Amitabha, Ivo Mihailovic, and Willy Zwaenepoel. "X-stream: Edge-centric graph processing using streaming partitions." In Proceedings of the Twenty-Fourth ACM Symposium on Operating Systems Principles, pp. 472-488. ACM, 2013.
\bibitem{fw:graphchi}
Kyrola, Aapo, Guy E. Blelloch, and Carlos Guestrin. "GraphChi: Large-Scale Graph Computation on Just a PC." In OSDI, vol. 12, pp. 31-46. 2012.
\bibitem{fw:myhadoop}
Krishnan, Sriram, Mahidhar Tatineni, and Chaitanya Baru. "myHadoop-Hadoop-on-Demand on Traditional HPC Resources." San Diego Supercomputer Center Technical Report TR-2011-2, University of California, San Diego (2011).
\bibitem{ssdhdd:moon}
Moon, Sangwhan, Jaehwan Lee, and Yang Suk Kee. "Introducing SSDs to the Hadoop MapReduce Framework." In Cloud Computing (CLOUD), 2014 IEEE 7th International Conference on, pp. 272-279. IEEE, 2014.
\bibitem{ssdhdd:tan}
Tan, Wei, Liana Fong, and Yanbin Liu. "Effectiveness Assessment of Solid-State Drive Used in Big Data Services." In Web Services (ICWS), 2014 IEEE International Conference on, pp. 393-400. IEEE, 2014.
\bibitem{ssdhdd:kang}
Kang, Yangwook, Yang-suk Kee, Ethan L. Miller, and Chanik Park. "Enabling cost-effective data processing with smart ssd." In Mass Storage Systems and Technologies (MSST), 2013 IEEE 29th Symposium on, pp. 1-12. IEEE, 2013.
\bibitem{ssdhdd:wu}
Wu, Dan, Wenhai Luo, Wenyan Xie, Xiaoheng Ji, Jian He, and Di Wu. "Understanding the Impacts of Solid-State Storage on the Hadoop Performance." In Advanced Cloud and Big Data (CBD), 2013 International Conference on, pp. 125-130. IEEE, 2013.
\bibitem{ssdhdd:ahn}
Ahn, Sungyong, Sangkyu Park, Jae-Ki Hong, and Wooseok Chang. "Performance Implications of SSDs in Virtualized Hadoop Clusters." In Big Data (BigData Congress), 2014 IEEE International Congress on, pp. 586-593. IEEE, 2014.
\bibitem{ethib:vienne}
Vienne, Jerome, Jitong Chen, Md Wasi-Ur-Rahman, Nusrat S. Islam, Hari Subramoni, and Dhabaleswar K. Panda. "Performance analysis and evaluation of infiniband fdr and 40gige roce on hpc and cloud computing systems." In High-Performance Interconnects (HOTI), 2012 IEEE 20th Annual Symposium on, pp. 48-55. IEEE, 2012.
\bibitem{ethib:yu}
Yu, Jie, Guangming Liu, Wei Hu, Wenrui Dong, and Weiwei Zhang. "Mechanisms of Optimizing MapReduce Framework on High Performance Computer." In High Performance Computing and Communications \& 2013 IEEE International Conference on Embedded and Ubiquitous Computing (HPCC\_EUC), 2013 IEEE 10th International Conference on, pp. 708-713. IEEE, 2013.
\bibitem{scaleupscaleout:appuswamy}
Appuswamy, Raja, Christos Gkantsidis, Dushyanth Narayanan, Orion Hodson, and Antony Rowstron. "Scale-up vs Scale-out for Hadoop: Time to rethink?." In Proceedings of the 4th annual Symposium on Cloud Computing, p. 20. ACM, 2013.
\bibitem{scaleupscaleout:michael}
Michael, Maged, Jose E. Moreira, Doron Shiloach, and Robert W. Wisniewski. "Scale-up x scale-out: A case study using nutch/lucene." In Parallel and Distributed Processing Symposium, 2007. IPDPS 2007. IEEE International, pp. 1-8. IEEE, 2007.
\bibitem{scaleupscaleout:chen}
Chen, Yanpei, Sara Alspaugh, and Randy Katz. "Interactive analytical processing in big data systems: A cross-industry study of mapreduce workloads." Proceedings of the VLDB Endowment 5, no. 12 (2012): 1802-1813.
\bibitem{bm:hibench}
Huang, Shengsheng, Jie Huang, Yan Liu, Lan Yi, and Jinquan Dai. "Hibench: A representative and comprehensive hadoop benchmark suite." In Proc. ICDE Workshops. 2010.
\bibitem{bm:graph500}
Murphy, Richard C., Kyle B. Wheeler, Brian W. Barrett, and James A. Ang. "Introducing the graph 500." Cray User’s Group (CUG) (2010).
\bibitem{cloudhpc:marathe}
Marathe, Aniruddha, Rachel Harris, David K. Lowenthal, Bronis R. de Supinski, Barry Rountree, Martin Schulz, and Xin Yuan. "A comparative study of high-performance computing on the cloud." In Proceedings of the 22nd international symposium on High-performance parallel and distributed computing, pp. 239-250. ACM, 2013.
\bibitem{bio:debruijngraph}
Pevzner, Pavel A., Haixu Tang, and Michael S. Waterman. "An Eulerian path approach to DNA fragment assembly." Proceedings of the National Academy of Sciences 98, no. 17 (2001): 9748-9753.
\bibitem{bio:quality1}
Medvedev, Paul, Eric Scott, Boyko Kakaradov, and Pavel Pevzner. "Error correction of high-throughput sequencing datasets with non-uniform coverage." Bioinformatics 27, no. 13 (2011): i137-i141.
\bibitem{bio:quality2}
Yang, Xiao, Sriram P. Chockalingam, and Srinivas Aluru. "A survey of error-correction methods for next-generation sequencing." Briefings in bioinformatics 14, no. 1 (2013): 56-66.
\bibitem{bio:gage}
Salzberg, Steven L., Adam M. Phillippy, Aleksey Zimin, Daniela Puiu, Tanja Magoc, Sergey Koren, Todd J. Treangen et al. "GAGE: A critical evaluation of genome assemblies and assembly algorithms." Genome research 22, no. 3 (2012): 557-567.
\bibitem{schadoop:fadika}
Fadika, Zacharia, Madhusudhan Govindaraju, Richard Canon, and Lavanya Ramakrishnan. "Evaluating hadoop for data-intensive scientific operations." In Cloud Computing (CLOUD), 2012 IEEE 5th International Conference on, pp. 67-74. IEEE, 2012.
\bibitem{schadoop:jha}
Jha, Shantenu, Judy Qiu, Andre Luckow, Pradeep Mantha, and Geoffrey C. Fox. "A tale of two data-intensive paradigms: Applications, abstractions, and architectures." In Big Data (BigData Congress), 2014 IEEE International Congress on, pp. 645-652. IEEE, 2014.
\bibitem{schadoop:matsunaga}
Matsunaga, Andréa, Maurício Tsugawa, and José Fortes. "Cloudblast: Combining mapreduce and virtualization on distributed resources for bioinformatics applications." In eScience, 2008. eScience'08. IEEE Fourth International Conference on, pp. 222-229. IEEE, 2008.
\bibitem{scgraph:guo}
Guo, Yong, Marcin Biczak, Ana Lucia Varbanescu, Alexandru Iosup, Claudio Martella, and Theodore L. Willke. "How well do graph-processing platforms perform? an empirical performance evaluation and analysis." In Parallel and Distributed Processing Symposium, 2014 IEEE 28th International, pp. 395-404. IEEE, 2014.
\bibitem{nsfcloud}
https://www.chameleoncloud.org/nsf-cloud-workshop/.
\end{thebibliography}

% that's all folks
\end{document}


